\documentclass[a4paper,12pt]{article}
\usepackage[utf8x]{inputenc}
\usepackage[LGR]{fontenc}
\usepackage{ucs}
\usepackage{amssymb} %Symbols

\usepackage{polyglossia}
%\setdefaultlanguage{greek}

\usepackage{listings}
\usepackage{textcomp}
\usepackage{color}

\setmainfont{Times New Roman}
\setsansfont{Arial}
\newfontfamily\greekfont{Times New Roman}
\setmainfont[Script=Greek]{Times New Roman}



\DeclareGraphicsExtensions{.pdf, .jpg}


%----------------------------------------------------------------------------------------
%	LISTINGS (CODE) TEMPLATE
%----------------------------------------------------------------------------------------

\lstset
{
	keywordstyle=\bfseries\ttfamily\color[rgb]{0,0,1},
	identifierstyle=\ttfamily,
	commentstyle=\color[rgb]{0.133,0.545,0.133},
	stringstyle=\ttfamily\color[rgb]{0.627,0.126,0.941},
	showstringspaces=false,
	basicstyle=\small,
	numberstyle=\footnotesize,
	numbers=left,
	stepnumber=1,
	numbersep=10pt,
	tabsize=2,
	breaklines=true,
	prebreak = \raisebox{0ex}[0ex][0ex]{\ensuremath{\hookleftarrow}},
	breakatwhitespace=false,
	aboveskip={1.5\baselineskip},
  	columns=fixed,
  	upquote=true,
  	extendedchars=true,
	frame=single
	inputencoding=utf8
}

\begin{document}

\begin{titlepage}

\newcommand{\HRule}{\rule{\linewidth}{0.5mm}} 

\center
 
%----------------------------------------------------------------------------------------
%	HEADING SECTION
%----------------------------------------------------------------------------------------

\textsc{\LARGE Εθνικό Μετσόβιο Πολυτεχνείο}\\[1.5cm] % Name of your university/college
\textsc{\Large Συστήματα και Τεχνολογίες Γνώσης}\\[0.5cm] % Major heading such as course name


%----------------------------------------------------------------------------------------
%	TITLE SECTION
%----------------------------------------------------------------------------------------

\HRule \\[0.4cm]
{ \huge \bfseries Μηχανική Μετάφραση Αριθμητικών  }\\[0.4cm]
\HRule \\[1.5cm]
 
%----------------------------------------------------------------------------------------
%	LOGO SECTION
%----------------------------------------------------------------------------------------

\includegraphics[scale=0.5]{ntua_logo} 
 
%----------------------------------------------------------------------------------------
%	AUTHOR SECTION
%----------------------------------------------------------------------------------------
\vfill

Ηλίας Φωτόπουλος \\ 03109106\\

%----------------------------------------------------------------------------------------

\end{titlepage}

\section{Στόχος}
Στόχος της άσκησης αποτελεί η σχεδίαση ενός συστήματος μηχανικής μετάφρασης, το οποίο θα δέχεται ως είσοδο έναν αριθμό από το 0 εώς και το 999 και θα παράγει ως έξοδο τον αριθμό σε ολογραφική μορφή. Η μετάφραση γίνεται από τα ελληνικά στα αγγλικά και αντιστρόφως. 
\section{Υλοποίηση}
Ο μηχανικός μεταφραστής χρησιμοποιεί τα μαθηματικά ως ενδιάμεσο επίπεδο μεταξύ των δύο γλωσσών (interlingua). Ουσιαστικά μετατρέπει ένα αλφαριθμητικό στην αριθμητική του απεικόνιση και στην συνέχεια μεταφράζει την αριθμητική απεικόνιση στην επιθυμητή γλώσσα.
		

\end{document}
