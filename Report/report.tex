\documentclass[a4paper,12pt]{article}
\usepackage[utf8x]{inputenc}
\usepackage[LGR]{fontenc}
\usepackage{ucs}
\usepackage{amssymb} %Symbols

\usepackage{polyglossia}
%\setdefaultlanguage{greek}

\usepackage{listings}
\usepackage{textcomp}
\usepackage{color}

\setmainfont{Times New Roman}
\setsansfont{Arial}
\newfontfamily\greekfont{Times New Roman}
\setmainfont[Script=Greek]{Times New Roman}



\DeclareGraphicsExtensions{.pdf, .jpg}


%----------------------------------------------------------------------------------------
%	LISTINGS (CODE) TEMPLATE
%----------------------------------------------------------------------------------------

\lstset
{
	keywordstyle=\bfseries\ttfamily\color[rgb]{0,0,1},
	identifierstyle=\ttfamily,
	commentstyle=\color[rgb]{0.133,0.545,0.133},
	stringstyle=\ttfamily\color[rgb]{0.627,0.126,0.941},
	showstringspaces=false,
	basicstyle=\small,
	numberstyle=\footnotesize,
	numbers=left,
	stepnumber=1,
	numbersep=10pt,
	tabsize=2,
	breaklines=true,
	prebreak = \raisebox{0ex}[0ex][0ex]{\ensuremath{\hookleftarrow}},
	breakatwhitespace=false,
	aboveskip={1.5\baselineskip},
  	columns=fixed,
  	upquote=true,
  	extendedchars=true,
	frame=single
	inputencoding=utf8
}

\begin{document}

\begin{titlepage}

\newcommand{\HRule}{\rule{\linewidth}{0.5mm}} 

\center
 
%----------------------------------------------------------------------------------------
%	HEADING SECTION
%----------------------------------------------------------------------------------------

\textsc{\LARGE Εθνικό Μετσόβιο Πολυτεχνείο}\\[1.5cm] % Name of your university/college
\textsc{\Large Συστήματα και Τεχνολογίες Γνώσης}\\[0.5cm] % Major heading such as course name


%----------------------------------------------------------------------------------------
%	TITLE SECTION
%----------------------------------------------------------------------------------------

\HRule \\[0.4cm]
{ \huge \bfseries Μηχανική Μετάφραση Αριθμητικών  }\\[0.4cm]
\HRule \\[1.5cm]
 
%----------------------------------------------------------------------------------------
%	LOGO SECTION
%----------------------------------------------------------------------------------------

\includegraphics[scale=0.5]{ntua_logo} 
 
%----------------------------------------------------------------------------------------
%	AUTHOR SECTION
%----------------------------------------------------------------------------------------
\vfill

Ηλίας Φωτόπουλος \\ 03109106\\

%----------------------------------------------------------------------------------------

\end{titlepage}

\section{Στόχος}
Στόχος της άσκησης αποτελεί η σχεδίαση ενός συστήματος μηχανικής μετάφρασης, το οποίο θα δέχεται ως είσοδο έναν αριθμό από το 0 εώς και το 999 και θα παράγει ως έξοδο τον αριθμό σε ολογραφική μορφή. Η μετάφραση γίνεται από τα ελληνικά στα αγγλικά και αντιστρόφως. 
\section{Υλοποίηση}
Ο μηχανικός μεταφραστής χρησιμοποιεί τα μαθηματικά ως ενδιάμεσο επίπεδο μεταξύ των δύο γλωσσών (interlingua). Ουσιαστικά μετατρέπει ένα αλφαριθμητικό στην αριθμητική του απεικόνιση και στην συνέχεια μεταφράζει την αριθμητική απεικόνιση στην επιθυμητή γλώσσα. Αναλυτικότερα η παραπάνω διαδικασία επιτυγχάνεται μέσω των παρακάτω μηχανισμών/συναρτήσεων:
	\subsection{Συναρτήσεις elnum/2 και ennum/2}
	Οι συναρτήσεις elnum/2 και ennum/2 χρησιμοποιούνται για να χωρίσουν τον ακέραιο σε εκατοντάδες δεκάδες και μονάδες. Αρχικώς κάθε συνάρτηση αντιμετωπίζει τις ειδικές περιπτώσεις αριθμών κάθε γλώσσας (για τα ελληνικά το 11 και 12 όπως φαίνεται παρακάτω):
\lstinputlisting[language=Prolog]{elnumspecial.pl}
Ενώ σε περίπτωση μη ειδικών αριθμών ακολουθείτε η διαδικασία που φαίνεται παρακάτω:
\lstinputlisting[language=Prolog]{elnum1000.pl}
Ανάλογη είναι και η διαδικασία που ακολουθεί η συνάρτηση ennum/2 η οποία έχει όμως διαφορετικούς ειδικούς αριθμούς.
	\subsection{DCG Rules}
	Ο επόμενος μηχανισμός στον οποίο στηρίζεται ο μηχανικός μεταφραστής είναι οι DCG Rules. Για τα ελληνικά οι κανόνες αυτοί φαίνονται παρακάτω:
\lstinputlisting[language=Prolog]{elrules.pl}
Ανάλογοι είναι και οι κανόνες για την Αγγλική γλώσσα.
	\subsection{Λεξικό (Βιβλιοθήκη DCG)}
	Για να επιτευχθεί η μηχανική μετάφραση χρειαζόμαστε και τα αντίστοιχα λεξικά για την Αγγλική και Ελληνική γλώσσα. Ενδεικτικά ένα μέρος του ελληνικού λεξικού φαίνεται παρακάτω:
	\lstinputlisting[language=Prolog]{eldic.pl}
	\subsection{Τελικές Συναρτήσεις}	 
	Οι τελικές συναρτήσεις που συνδυάζουν όλα τα παραπάνω και πετυχαίνουν την τελική μετάφραση φαίνονται παρακάτω:
	\lstinputlisting[language=Prolog]{mainfunctions.pl}
	
	Το επιθυμητό αποτέλεσμα επιτυγχάνεται μέσω των συναρτήσεων translategreek/2 και translateenglish/2. Όπως περιγράψαμε αρχικώς η συνάρτηση translategreek/2 σε πρώτο στάδιο μεταφράζει τα ελληνικά σε αριθμητική μορφή (interlingua στάδιο) και στην συνέχεια μεταφράζει τον αριθμό στα Αγγλικά στην ολογραφική του μορφή. Ενώ η συνάρτηση translateenglish/2 δουλεύει ακριβώς με τον ίδιο τρόπο απλά επιτελεί την αντίστροφη διαδικασία.
\section{Αμφισημία}
Κατά την ανάπτυξη του μηχανικού μεταφραστή υπήρξαν και προβλήματα αμφισημίας όπως τα παρακάτω.
\begin{itemize}
  \item Στα αγγλικά χρειάζονται δύο λέξεις για την περιγραφή των εκατοντάδων (one hundred - εκάτο)
  \item Στα ελληνικά είχαμε τις ειδικές περιπτώσεις του 11 και 12, όπου δεν μπορούσε να ισχύσει ο γενικός κανόνας 11 --> Δέκα ένα 
  \item Όπως αναφέρεται και στην εκφώνηση έχουμε πρόβλημα για το 100 (εκατό ή εκατόν), το οποίο λύθηκε με ειδικό χειρισμό και στο λεξικό και στους κανόνες DCG.
\end{itemize}

\end{document}
